\documentclass[a4paper,12pt]{article}
\pagestyle{empty}
\usepackage{sty}
\usepackage{listings}

\begin{document}
\pagecolor{background}
\color{foreground}

%date
\par{
	\flushleft\large
	16 MAy, 2024
}

%title
\par{\Huge
	\flushleft\usefont{T1}{qzc}{m}{n}
	Documentation for the Computer Architecture Project
}

%line
\line

%body
\section{Introduction}
	\par{
		This document is a report for the Computer Architecture (CA) project.
		The project aims to simulate a fictional processor design and
		architecture using C. The project at hand was one of four project
		packages offered. The project packet of intrest is the Double McHarvard
		with cheese circular shifts.
	}
	\par{
		This package is using a Harvard architecture with one data bus and one
		instruction bus. The processor has a \tt{8-bit} data bus and a
		\tt{16-bit} instruction bus, while all the registers are \tt{8-bits}.
		To elaborate, the processor has access to \tt{64} General Purpose
		registers, a single status register, and a single program counter.
		Furthermore, the processor has a \tt{8-bit} ALU. 
	}

\section{Methodology}
	\par{
		The project was implemented using \tt{C} and was designed to be as
		modular as possible. To expand on this, the entire project consists of
		multiple header files, each containing a specific part of the project,
		and a single \tt{main.c} file that contains the main function. The
		project was divided into three main parts, which will be explained in
		detail below:

		\ls{
			The Memory Architecture.
		}\ls{
			The Instruction Set Architecture.
		}\ls{
			The Data Path.
		}
	}
	\subsection{Memory Architecture}
		\par{
			The memory architecture was further divided into three parts to
			ensure modularity. Each part was implemented in a separate header
			file. A brief description will be provided below:
			\ls{
				The Data Memory.
			}\ls{
				The Instruction Memory.
			}\ls{
				The Register File.
			}
		}
		\subsubsection{Data Memory}
			\par{
				The data memory was implemented using a pointer to an array of
				\tt{8-bit} integers. To be more specific, the data type of the
				pointer was \tt{uint8\_t}. The data memory was implemented in the
				\tt{data\_memory.h} header file. Below is a brief description of
				the functions implemented in the \tt{data\_memory.h} header file:
			}

			\newpage

			\ls{
				Init Data Memory: This function `callocs' memory for the data
				memory and returns a pointer to the allocated memory. The function
				uses the \tt{calloc} function instead of the \tt{malloc} function
				to ensure that the allocated memory is initialized to zero.
			}\ls{
				Kill Data Memory: This function `frees' the memory allocated for
				the data memory. The function takes a pointer to the data memory
				as an argument and `frees' the memory allocated for the data
				memory.
			}\ls{
				Set Data Memory: This function sets the data at a specific address
				in the data memory. The function takes a pointer to the data memory,
				the address of the data to be set, and the data to be set as
				arguments.
			}\ls{
				Get Data Memory: This function returns the data at a specific
				address in the data memory. The function takes a pointer to the data
				memory and the address of the data to be retrieved as arguments.
			}\ls{
				Copy Data Memory: This function copies the data memory. The function
				takes a pointer to the data memory as an argument and returns a
				pointer to the copied data memory using deep copy.
			}\ls{
				Pretty Print Data Memory: This function prints the data memory.
				The function takes a pointer to the data memory as an argument
				and prints the data memory in a human-readable format. For all
				consecutive addresses with \tt{0x00} data, the function prints
				the address range instead of printing each address
				individually.
			}\ls{
				Pretty Print Diff Data Memory: This function prints the difference
				between two data memories. The function takes two pointers to the
				data memories as arguments and prints the difference between the
				two data memories in a human-readable format. If the data at a
				specific address is different between the two data memories, the
				function prints the address and the old and new data. If there is
				no difference between the two data memories, the function prints
				`No Difference'.
			}

		\subsubsection{Instruction Memory}
			\par{
				The instruction memory was implemented using a pointer to an array of
				\tt{16-bit} integers. To be more specific, the data type of the pointer
				was \tt{uint16\_t}. The instruction memory was implemented in the
				\tt{instruction\_memory.h} header file. For the sake of brevity, the
				functions implemented in the \tt{instruction\_memory.h} header file
				will not be described in detail as they are identical to the functions
				implemented in the \tt{data\_memory.h} header file with the exception
				of the data type of the pointer, and the lack thereof of the
				\emph{Pretty Print Diff Data Memory} function.
			}
	
		\newpage

		\subsubsection{Register File}
			\par{
				The registers were implemented using a pointer to a structure that
				contained an array of \tt{8-bit} integers for the general-purpose
				registers, an \tt{8-bit} integer for the status register, and an
				\emph{8-bit} integer for the program counter. The register file was
				implemented in the \tt{register\_file.h} header file.
				
				To be as specific as possible, the structure used to represent the
				registers was defined as follows:

				\begin{lstlisting}
					typedef struct {
						uint8_t general_purpose_registers[64];
						uint8_t status_register;
						uint8_t program_counter;
					} RegisterFile;
				\end{lstlisting}

				Finally below is a brief description of the functions implemented in
				the \tt{register\_file.h} header file:
			}
			\ls{
				Init Register File: This function initializes the register file, and
				returns a pointer to the allocated memory. This function is identical
				to the \emph{Init Data Memory} function with the exception of the data
				type of the pointer.
			}\ls{
				Kill Register File: This function `frees' the memory allocated for the
				register file. This function is identical to the \emph{Kill Data Memory}
				function with the exception of the data type of the pointer.
			}\ls{
				Set Status Register flag: This function sets a specific flag in the
				status register. The function takes a pointer to the register file, a
				character representing the flag to be set, and a boolean representing
				the value to set the flag to as arguments.
			}\ls{
				Get Status Register flag: This function returns the value of a specific
				flag in the status register. The function takes a pointer to the register
				file and a character representing the flag to be retrieved as arguments.
			}\ls{
				Copy Register File: This function copies the register file. This function
				is identical to the \emph{Copy Data Memory} function with the exception of
				the data type of the pointer.
			}\ls{
				Pretty Print Register File: This function prints the register file. This
				function is identical to the \emph{Pretty Print Data Memory} function with
				the exception of the data type of the pointer.
			}
	
	\subsection{Instruction Set Architecture}
		\par{
			The instruction set architecture was implemented using a set of header
			files, each containing a specific stage in the execution of an instruction.
			A brief description of each stage will be provided below:
			\ls{
				The Fetch Stage.
			}\ls{
				The Decode Stage.
			}\ls{
				The Execute Stage.
			}\ls{
				The Assembly Code Parser*
			}
		}
		\subsubsection{Fetch Stage}
			\par{
				The fetch stage was implemented in the \tt{fetch.h} header file. The
				fetch stage is responsible for fetching the instruction from the
				instruction memory and incrementing the program counter. 
			}
		
		\subsubsection{Decode Stage}
			\par{
				The decode stage was implemented in the \tt{decode.h} header
				file. The decode stage is responsible for decoding the
				instruction and extracting the opcode and operands and creating
				a structure to represent the instruction which will be denoted
				as a \bf{Process Control Block} (PCB). The PCB was defined as
				follows:
				\begin{lstlisting}
					typedef struct {
						uint8_t opcode; // 4 bits
						uint8_t operand1; // 6 bits
						uint8_t operand2; // 6 bits
						uint8_t R1; // 8 bits
						uint8_t R2; // 8 bits
						uint8_t IMM; // 8 bits
						bool is_R_format; // 1 bit
					} PCB;
				\end{lstlisting}
			}
			\par{
				The decode stage had two functions: the \tt{decode} function and
				the \tt{pretty\_print\_pcb} function. The \tt{decode} function
				takes a pointer to the instruction memory and a pointer to the
				register file as arguments and returns a pointer to the PCB. The
				\tt{pretty\_print\_pcb} function takes a pointer to the PCB as an
				argument and prints the PCB in a human-readable format.
			}

		\subsubsection{Execute Stage}
			\par{
				The execute stage was implemented in the \tt{execute.h} header file.
				The execute stage is responsible for executing the instruction
				represented by the PCB. The execute stage had a single function
				called \tt{execute} which takes a pointer to the PCB and a pointer
				to the register file as arguments and returns nothing.
			}
			\par{
				The execute function was implemented using a switch statement that
				switches on the opcode of the PCB. The switch statement calls the
				appropriate function based on the opcode of the PCB, and handles
				the updating of the status register if necessary.
			}

		\subsubsection{Assembly Code Parser}
			\par{
				This is the only header file that is located in the \tt{ISA}
				directory but is not part of the instruction set architecture.
				The assembly code parser was implemented in the \tt{parser\_asm.h}
				header file. This header file is responsible for parsing the
				assembly code line by line. For each line of assembly code, the
				parser tokenizes the line and converts the tokens to an instruction
				in the form of a \tt{uint16\_t} integer. The parser then writes the
				instruction to the instruction memory. Finally, as the parser reaches
				the \tt{EOF} of the assembly code file, it appends to the instruction
				memory a \tt{0xf000} instruction to signify the end of the program and
				a halt instruction. Once the parser has finished parsing the assembly
				code, it returns a pointer to the instruction memory.
			}
	
	\subsection{Data Path}
		\par{
			The data path was implemented in the \tt{main.c} file. The data path
			is responsible for fetching the instruction from the instruction memory,
			decoding the instruction, executing the instruction, and updating the
			register file and data memory accordingly.
		}
		\par{
			The main function first initializes the data memory, instruction memory,
			and register file. The main function then calls the assembly code parser
			to parse the assembly code and write the instructions to the instruction
			memory. Then the main function enters the main loop of the data path, which will
			be described in detail in the next paragraph. Finally, the main function
			frees the memory allocated for the data memory, instruction memory, and
			register file, And prints the final state of the data memory and register
			file.
		}
		\par{
			The main loop of the data path consists of three stages: the fetch stage,
			the decode stage, and the execute stage. All these stages are occurring
			within the same clock cycle. The main loop keeps track which instruction
			should be fetched, decoded, or executed by using the program counter and
			deviations from it. Thus it would take exaclty $3 + (n - 1)$ clock cycles
			to execute the program where $n$ is the number of instructions in the
			assembly code. This holds if there are no jumps or branches in the
			assembly code.
		}
		\par{
			To elaborate more on the algorithm that made this level of concurrency
			possible, the main loop has within it a \\{i} variable that is firstly
			initialized to zero. With every iteration of the main loop, the \tt{i}
			variable is incremented by one. The \tt{i} variable is then used to
			determine whether a decode or execute stage should be executed this
			iteration or not. The algorithm is as follows:
		}
		\cd{The concurrency algorithm}{Instructions $\in \bb Z^{max}$}{\tt{NULL}}{
			\tt{i}, \tt{HALT} $\gets 0$ \\
			\tt{input\_for\_decoder} $\gets$ \tt{NULL} \\
			\tt{input\_for\_executor} $\gets$ \tt{NULL} \\
			\While{\tt{HALT} $\neq 1$}{
				\If{\tt i $\geq 2$}{
					execute(\tt{input\_for\_executor}) \\
					\If{\tt{input\_for\_executor} $\to$ opcode $= \tt{0xF}$}{
						\tt{HALT} $\gets 1$ \\
					}

					\If{\tt{input\_for\_executor} $\to$ opcode $= \tt{BRANCH}$}{
						\tt i $\gets 0$ \\
						\tt{PC} $\gets$ \tt{input\_for\_executor} $\to$ IMM \\
					}
				}
				\If{\tt i $\geq 1$}{
					\tt{input\_for\_executor} $\gets$ decode(\tt{input\_for\_decoder}) \\
				}
				\If{\tt i $\geq 0$}{
					\tt{input\_for\_decoder} $\gets$ fetch\_instruction(\tt{PC}) \\
				}
			}
		}
		\par{
			As can be seen from the algorithm above, whenever a branch instruction
			is encountered, the \tt i variable is reset to zero and the program
			counter is updated to the immediate value of the branch instruction.
			Furthermore, the algorithm also checks for the \tt{HALT} instruction
			and terminates the program if it is encountered.
		}


\section{Results}
	\par{
		The project was tested using a set of assembly code files that were
		designed to test the functionality of the processor. The first test
		was a simple addition test that added two numbers and stored the result
		in a register. The second test was a factorial test that calculated the
		factorial of a number and stored the result in memory. The third test
		was a cumulative sum test that calculated the cumulative sum of a number
		and stored the result in memory.
	}
	\par{
		The first test was designed to insure that the processor did manage to
		do the assembly code with in the estimated number of clock cycles. The
		second test and the third test were designed to test the logic of the
		processor, in particular the branching and jumping logic. The tests were
		successful and the processor was able to execute the assembly code
		correctly.
	}

\end{document}